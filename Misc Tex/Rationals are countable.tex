
Let the rational  numbers $\mathbb{Q}$ be constructed as follows:
\begin{table}[h]
    \centering
    \begin{tabular}{|c|c|c|c|c|}
        \hline
        & 1 & 2 & 3 & ...\\ \hline
        1 & $\frac{1}{1}$   & $\frac{2}{1}$   & $\frac{3}{1}$ & ...  \\ \hline
        2 & $\frac{1}{2}$   & $\frac{2}{2}$   & $\frac{3}{2}$  & ...   \\ \hline
        3 & $\frac{1}{3}$  & $\frac{2}{3}$  & $\frac{3}{3}$ & ... \\ \hline
        ... & ...  & ...& ...& ... \\ \hline
    \end{tabular}
    \label{tab:placeholder_label}
    \caption{\cite{suber1998crash}}
\end{table}

It follows then that by using the path indicated below, you could construct a set $\mathbb{Q} = \{ \frac{1}{1}, \frac{2}{1}, \frac{1}{2}, \frac{3}{1}, \frac{2}{2}, \frac{1}{3}, ...\}$
in this way you can align each member to a corresponding member of $\mathbb{N}$ 

Similarly, for multiplication of $\mathbb{N} \times \mathbb{N}$ giving a cardinality of $\lvert{\mathbb{N}}\rvert \cdot \lvert{\mathbb{N}}\rvert$ you can construct a countable set in the same way, replacing denominators with the second factor of the number in its row.





